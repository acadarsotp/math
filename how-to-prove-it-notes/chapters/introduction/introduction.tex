\chapter{Introduction}

\section{Prime Numbers}
This is a book about deductive reasoning in mathematics and mathematical proofs. 
Fundamental theorems regarding prime numbers are introduced in this chapter to provide an idea of what to expect later in the book.

\dfn{Prime Numbers}{A natural number $n>1$ is considered prime if and only if its only positive divisors are  $1$ and  $p$.}

As a first proof, let's test the conjecture that there are infinitely many prime numbers.
\thm{Euclid's Second Second Theorem}{There are infinitely many prime numbers.}{}

\begin{myproof}
If we assume that there is a finite list of prime numbers denoted as $p_1, p_2, \ldots, p_n$, we can create a number $m$ as follows:

\[
m = p_1p_2 \ldots p_n + 1
\]

Note that $m$ is not divisible by any of the primes $p_n$.
\footnote{This is because the remainder when dividing $m$ by any of the primes $p_n$ is $1$, 
which means $m$ is not divisible by them.}
We acknowledge the fact that every integer larger than 1 is either a prime or can be written as a product of two or more primes. 
If $m$ is a prime, it can't be on the list because that would contradict the first assumption that all primes are on the list. 
Now, assume $m$ is a product of primes. 
Take $q$ as one of the primes in that product and an element on the list, so $m$ is divisible by $q$. 
Then again, we reach a contradiction because $m$ is not divisible by any $p_n$.

Since the assumption that there is a finite amount of prime numbers reached a contradiction, there must be an infinite amount of prime numbers.
\end{myproof}

\thm{Mersenne Composites}{
  For every composite integer $n > 1$, $2^n - 1$ will also be composite.
  \footnote{The analogy is sometimes true for primes, these are called Mersenne primes.}
}{thm:_mersenne_composites}

\begin{myproof}
  If $n$ is composite then it can be expressed as  $n = ab$. 
  Let  $x = 2^b -1$ and  $y = 1 + 2^b + 2^{2b} + \ldots + 2^{(a-1)b}$.
  Then:
\begin{align*}
  xy &= (2^b - 1)*(1 + 2^b + 2^{2b} + \ldots + 2^{(a-1)b})\\
     &= 2^b * (1 + 2^b + 2^{2b} + \ldots + 2^{(a-1)b}) -  (1 + 2^b + 2^{2b} + \ldots + 2^{(a-1)b})\\
     &= (2^b + 2^{2b} + 2^{3b} + \ldots + 2^{ab}) -  (1 + 2^b + 2^{2b} + \ldots + 2^{(a-1)b})\\
     &= 2^{ab} - 1\\
     &= 2^n - 1.  
\end{align*}
\end{myproof}
\section{Exercises}

\qs{}{
  *1. (a) Factor $2^n - 1 = 32,767$ into a product of two smaller positive integers.

  *1. (b) Find an integer $x$ such that $1 < x < 2^{32,767} - 1$ and $2^{32,767} - 1$ is divisible by $x$.
}

\sol \textbf{(a)} $n = log_2(32,767+1) = 15$, we know that if $n$ is not prime then $2^n - 1$ is not prime, so 32,767 is a composite number. 
One possible solution is $4,681 * 7 = 32,767$.

\sol \textbf{(b)} $2^{32,767} - 1$ is composite so there exist at least two $x$ that can solve the problem.
From the proof of \textit{Conjecture 2}, if $n$ is not prime then $n = a*b$.
Also $2^n - 1 = (2^b - 1) * (1 + 2^b + 2^{2b} + \ldots + 2^{(a-1)b})$.
So one possibility is $x = 2^7 - 1$.

\qs{}{
  2. Make some conjectures about the values of $n$ for which $3^n - 1$ is prime or the values of $n$ for which $3^n - 2^n$ is prime.
  (You might start by making a table similar to \textit{Figure I.1})
}


