\section{Exercises}

\qs{}{
  *1. (a) Factor $2^n - 1 = 32,767$ into a product of two smaller positive integers.

  *1. (b) Find an integer $x$ such that $1 < x < 2^{32,767} - 1$ and $2^{32,767} - 1$ is divisible by $x$.
}

\sol \textbf{(a)} $n = log_2(32,767+1) = 15$, we know that if $n$ is not prime then $2^n - 1$ is not prime, so 32,767 is a composite number. 
One possible solution is $4,681 * 7 = 32,767$.

\sol \textbf{(b)} $2^{32,767} - 1$ is composite so there exist at least two $x$ that can solve the problem.
From \ref{label}, if $n$ is not prime then $n = a*b$.
Also $2^n - 1 = (2^b - 1) * (1 + 2^b + 2^{2b} + \ldots + 2^{(a-1)b})$.
So one possibility is $x = 2^7 - 1$.

\qs{}{
  2. Make some conjectures about the values of $n$ for which $3^n - 1$ is prime or the values of $n$ for which $3^n - 2^n$ is prime.
  (You might start by making a table similar to \textit{Figure I.1})
}

\sol \begin{table}[h]
  \centering
  \label{tab:3n_minus_1}
  \begin{tabular}{|c|c|}
    \hline
    $n$ & $3^n - 1$\\
    \hline
    1 & 2\\
    2 & 8\\
    3 & 26\\
    4 & 80\\
    5 & 242\\
    6 & 728\\
    7 & 2,186\\
    8 & 6,560\\
    9 & 19,682\\
    10 & 59,408\\
    \hline
  \end{tabular}
  \caption{Values of $3^n - 1$ fom 1 to 10}
\end{table}

\con{}{For every positive integer $n$,  $3^n - 1$  is even.}

\begin{table}[h]
  \centering
  \label{tab:3n_minus_2n}
  \begin{tabular}{|c|c|}
    \hline
    $n$ & $3^n - 2^n$\\
    \hline
    1 & 0\\
    2 & 1\\
    3 & 5\\
    4 & 19\\
    5 & 65\\
    6 & 211\\
    7 & 665\\
    8 & 2,059\\
    9 & 6,305\\
    10 & 19,171\\
    \hline
  \end{tabular}
  \caption{Values of $3^n - 2^n$ fom 1 to 10}
\end{table}

\con{}{For every positive integer $n>1$,  $3^n - 2^n$  is odd.}
There are no aparent relationships involving primes.  

\qs{}{
 *3. The proof of Theorem 3 gives a method for finding a prime number different from any in a given list of prime numbers.

 (a) Use this method to find a prime different from 2, 3, 5, and 7.

 (b) Use this method to find a prime different from 2, 5, and 11. 
}

\sol \textbf{(a)} Given said proof, we have a list of primes $p_1,p_2 \ldots p_n$.
We can create a number $m = p_1 * p_2 * \ldots * p_n + 1$, for the given list this results in $m = 211$.

\sol \textbf{(b)} With the same process we reach $m = 111$.

\qs{}{4. Find five consecutive integers that are not prime.}

\sol 
We can solve this with a simple python script:
\setminted[python]{breaklines, framesep=2mm, fontsize=\footnotesize, numbersep=5pt}
\begin{minted}{python}
  L = []
  for n in range (2, 100):
      prime = True

      for i in range (2, int(n/2+1)):
          if n%i == 0:
              prime = False
              break

      if prime:
          L = []
      else:
          L.append(n)

      if len(L) == 5:
          print(L)
          break
\end{minted}
Wich outputs: $24, 25, 26, 27, 28$.
 
\qs{}{5. Use the table in Figure I.1 and the discussion on p. 5 to find two more perfect numbers.}

\sol We know that if $2^n - 1$ is prime, then $2^{n-1}(2^n -1)$ is perfect.
For $n = 7$, $2^n - 1$ is prime so $2^{n-1}(2^n -1) = 8,128$ is perfect.
For $n = 5$, $2^n - 1$ is prime so $2^{n-1}(2^n -1) = 496$ is perfect.

\qs{}{
  6. The sequence 3, 5, 7 is a list of three prime numbers such that each pair of adjacent numbers in the list differ by two.
  Are there any more such “triplet primes”?
}
\newpage

\sol Again, this can be solved using python:
\setminted[python]{breaklines, framesep=2mm, fontsize=\footnotesize, numbersep=5pt}
\begin{minted}{python}
  L = []
  for n in range (2, 10000):
      prime = True

      for i in range (2, int(n/2+1)):
          if n%i == 0:
              prime = False
              break

      if prime:
          if len(L) == 0:
              L.append(n)
          else:
              if n == L[-1]+2:
                  L.append(n)
              else:
                  L = [n]

      if len(L) == 3:
          print(L)
          L.pop(0)
\end{minted}
The script only prints the sequence $3, 5, 7$, so it seems like there aren't any more such "triplet primes".
\footnote{
  According to number theory, prime triplets are of the form $(p, p+2, p+6)$ or $(p, p+4, p+6)$.
  Wich differ from the definition given by the exercise.
}

\qs{}{
  7. A pair of distinct positive integers (m, n) is called amicable if the sum of all positive integers smaller than n that divide n is m,
  and the sum of all positive integers smaller than m that divide m is n.
  Show that (220, 284) is amicable. 
}

\sol With python:
\setminted[python]{breaklines, framesep=2mm, fontsize=\footnotesize, numbersep=5pt}
\begin{minted}{python}
  def find_divisors(number):
      divisors = []
      for i in range(1, number):
          if number % i == 0:
              divisors.append(i)
      return divisors

  def check_amicable(m,n):
      divm = find_divisors(m)
      divn = find_divisors(n)

      if sum(divm) == n and sum(divn) == m:
          return True
      else:
          return False

  print(check_amicable(220,284))
\end{minted}
The output of the script shows that this pair is indeed amicable.
