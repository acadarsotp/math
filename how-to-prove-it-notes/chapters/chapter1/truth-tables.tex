\section{Truth Tables}

\dfn{Truth Table}{A Truth Table summarizes all possibilities for a given logic statement.}
\begin{center}
 \begin{tabular}{|c|c|c|}
   \hline
   $P$ & $Q$ & $P \land Q$\\
   \hline
   F & F & F\\
   F & T & F\\
   T & F & F\\
   T & T & T\\
   \hline
 \end{tabular}
 \captionof{table}{$P \land Q$}
 \label{tab:P_and_Q}
\end{center}


\nt{Unless specified otherwise, in mathematics $\lor$ is inclusive.}

\ex{Truth Table}{Make a truth table for the formula $\neg (P \land Q) \lor \neg R$.}
\sol 
\begin{table}[h]
  \centering
  \label{tab:not_(P_and_Q)_or_not_R }
  \begin{tabular}{|c|c|c|c|c|c|c|}
    \hline
    $P$ & $Q$ & $R$ & $P \land Q$ & $\neg (P \land Q)$ & $\neg R$ & $\neg (P \land Q) \lor \neg R$\\
    \hline 
    F & F & F & F & T & T & T\\      
    F & F & T & F & T & F & T\\
    F & T & F & F & T & T & T\\
    F & T & T & F & T & F & T\\
    T & F & F & F & T & T & T\\
    T & F & T & F & T & F & T\\
    T & T & F & T & F & T & T\\ 
    T & T & T & T & F & F & F\\
    \hline
  \end{tabular}
  \caption{$\neg (P \land Q) \lor \neg R$}
\end{table}

Truth tables are to be used to study if an argument is valid.
Consider again the following argument:

\begin{tabular}{l}
It will either rain or snow tomorrow.\\
It's too warm for snow.\\
Therefore, it will rain.\\
\end{tabular}

This argument is of the form: 
\begin{align*}
    &P \lor Q\\
    &\neg Q\\
          \noalign{\vspace{-8pt}}
          \cline{1-2}
          \noalign{\vspace{-4pt}}
    &\therefore P\\
\end{align*}

\newpage

So we can build the following truth table:

\begin{table}[h]
\centering
\label{tab:snow_or_rain }
\begin{tabular}{|c|c|c|c|c|}
  \hline
  $P$ & $Q$ & $P \lor Q$ & $\neg Q$ & $P$\\
  \hline 
  F & F & F & T & F\\ 
  F & T & T & F & F\\  
  T & F & T & T & T\\  
  T & T & T & F & T\\
  \hline
\end{tabular}
\caption{Rain or Snow Tomorrow}
\end{table}

On the previous section a valid argument was defined as one in which the conclusion is true if all premises are true. 
For this argument, the second to last row of the table represents this scenario. 
We can see that if both premises are true then the conclusion is true, therefore this is a valid argument. 

\ex{Determine whether the following arguments are valid}{
  \begin{enumerate}
    \item Either John isn't smart and he is lucky, or he’s smart.\\
          John is smart.\\
          Therefore, John isn't lucky.\\ 
    \item The butler and the cook are not both innocent.\\
          Either the butler is lying or the cook is innocent.\\
          Therefore, the butler is either lying or guilty.\\
  \end{enumerate}
}

\sol \textbf{1.} 
  \begin{align*}
      &(\neg P \land Q) \lor P\\
      &P\\
            \noalign{\vspace{-8pt}}
            \cline{1-2}
            \noalign{\vspace{-4pt}}
      &\therefore \neg Q\\
  \end{align*}

  The reasoning seems faulted, lets build the truth table:

  \begin{table}[h]
  \centering
  \label{tab:smart_and_lucky }
  \begin{tabular}{|c|c|c|c|c|}
    \hline
    $P$ & $Q$ &  $(\neg P \land Q) \lor P$ & $P$ & $\neg Q$\\
    \hline 
    F & F & F & F & T\\   
    F & T & T & F & F\\
    T & F & T & T & T\\
    T & T & T & T & F\\ 
    \hline
  \end{tabular}
  \caption{John is Smart and Lucky, or just smart}
\end{table}

The last row shows a case in which both premises are true but the conclusion is false, so the argument is not valid.

\sol \textbf{2.} 
  \begin{align*}
      &\neg (B_i \land C_i)\\
      &B_l \lor C_i\\
            \noalign{\vspace{-8pt}}
            \cline{1-2}
            \noalign{\vspace{-4pt}}
      &\therefore \neg B_i \lor B_l\\
  \end{align*}

  \newpage

  \begin{table}[h]
  \centering
  \label{tab:lying_or_innocent }
  \begin{tabular}{|c|c|c|c|c|c|}
    \hline
    $B_i$ & $C_i$ & $B_l$ & $\neg (B_i \land C_i)$ & $B_l \lor C_i$ & $\neg B_i \lor B_l$\\ 
    \hline 
    F & F & F & T & F & T\\ 
    F & F & T & T & T & T\\ 
    F & T & F & T & T & T\\ 
    F & T & T & T & T & T\\   
    T & F & F & T & F & F\\
    T & F & T & T & T & T\\
    T & T & F & F & T & F\\
    T & T & T & F & T & T\\    
    \hline
  \end{tabular}
  \caption{The Butler is Lying or Innocent}
\end{table}

The argument is valid.

\dfn{Equivalent Formulas}{Equivalent formulas always have the same truth values.}

\begin{center} 
  \begin{tabular}{|c|c|c|c|}
    \hline
    $P$ & $Q$ & $\neg (P \land Q)$ & $\neg P \lor \neg Q$ \\
    \hline 
    F & F & T & T\\   
    F & T & T & T\\
    T & F & T & T\\
    T & T & F & F\\
    \hline
  \end{tabular}
  \captionof{table}{Equivalent Formulas}
\end{center}

The following is a list of equivalences that come up often in logic.

\axm{De Morgan's Laws}{
    \centering
    $\neg (P \lor Q)$ is equivalent to $\neg P \land \neg Q$.\\ 
    $\neg (P \land Q)$ is equivalent to $\neg P \lor \neg Q$. 
}{}

\axm{Commutative Laws}{
    \centering
    $P \land Q$ is equivalent to $Q \land P$.\\
    $P \lor Q$ is equivalent to $Q \lor P$.
}{}

\axm{Associative Laws}{
    \centering
    $P \land (Q \land R)$ is equivalent to $(P \land Q) \land R$.\\
    $P \lor (Q \lor R)$ is equivalent to $(P \lor Q) \lor R$.
}{}

\axm{Idempotent Laws}{
    \centering
    $P \land P$ is equivalent to $P$.\\
    $P \lor P$ is equivalent to $P$.
}{}

\axm{Distributive Laws}{
    \centering
    $P \land (Q \lor R)$ is equivalent to $(P \land Q) \lor (P \land R)$.\\
    $P \lor (Q \land R)$ is equivalent to $(P \lor Q) \land (P \lor R)$.
}{}

\axm{Absorption Laws}{
    \centering
    $P \lor (P \land Q)$ is equivalent to $P$.\\
    $P \land (P \lor Q)$ is equivalent to $P$.
}{}

\axm{Double Negation Law}{
    \centering
    $\neg \neg P$ is equivalent to $P$.
}{}

\ex{Find simpler formulas equivalent to these formulas}{
  \begin{enumerate}
    \item $\neg (P \lor \neg Q)$.
    \item $\neg (Q \land \neg P) \lor P$.
  \end{enumerate}
}

\sol \textbf{1.} $\neg (P \lor \neg Q)$ is equivalent to $\neg P \land Q$.\\
\sol \textbf{2.} $\neg (Q \land \neg P)$ is equivalent to $\neg Q \lor P$.
We have $(\neg Q \lor P) \lor P$ which is equivalent to $\neg Q \lor P$.

\dfn{Tautology}{A Tautology is a formula which is always true.}
\dfn{Contradiction}{A Contradiction is a formula which is always false.}

\axm{Tautology Laws}{
    \centering
    $P \land$(a tautology) is equivalent to $P$.\\
    $P \lor$(a tautology) is a tautology.\\
    $\neg$(a tautology) is a contradiction.     
}{}

\axm{Contradiction Laws}{
    \centering
    $P \land$(a contradiction) is a contradiction.\\
    $P \lor$(a contradiction) is equivalent to $P$.\\
    $\neg$(a contradiction) is a tautology.
}{}

\ex{Find simpler formulas equivalent to these formulas}{
  \begin{enumerate}
    \item $P \lor (Q \land \neg P)$.
    \item $\neg (P \lor (Q \land \neg R)) \land Q$.
  \end{enumerate}
}

\sol \textbf{1.} 
$P \lor (Q \land \neg P)$ is equivalent to $(P \lor Q) \land (P \lor \neg P)$, which simplifies to $P \lor Q$.\\
\sol \textbf{2.}
\begin{align*}
  \intertext{We start by simplifying the first part of the statement}
  &\neg(P \lor (Q \land \neg R)) \\
  &\neg P \land \neg(Q \land \neg R) \\
  &\neg P \land (\neg Q \lor R) \\
  \intertext{Then adding the second part}
  &\neg P \land (\neg Q \lor R) \land Q \\
  &\neg P \land (\neg Q \land Q) \lor (R \land Q) \\
  &\neg P \land (R \land Q)
\end{align*}

\subsection{Exercises}

\qs{}{
  3. In this exercise we will use the symbol + to mean exclusive or. 
  In other words, P + Q means “P or Q, but not both.”

  (a) Make a truth table for $P + Q$.

  (b) Find a formula using only the connectives $\land$, $\lor$, and $\neg$ that is
  equivalent to $P + Q$. Justify your answer with a truth table.
}

\sol \textbf{(a)} 
  \begin{table}[h]
  \centering
  \label{tab:ex3-1 }
  \begin{tabular}{|c|c|c|}
    \hline
    $P$ & $Q$ & $P + Q$\\ 
    \hline 
    $F$ & $F$ & $F$\\   
    $F$ & $T$ & $T$\\
    $T$ & $F$ & $T$\\
    $T$ & $T$ & $F$\\
    \hline
  \end{tabular}
  \caption{Exercise 3.1}
\end{table}

\sol \textbf{(b)} $(P \land \neg Q) \lor (\neg P \land Q)$ 
  \begin{table}[h]
  \centering
  \label{tab:ex3-2 }
  \begin{tabular}{|c|c|c|}
    \hline
    $P$ & $Q$ & $(P \land \neg Q) \lor (\neg P \land Q)$\\ 
    \hline 
    $F$ & $F$ & $F$\\   
    $F$ & $T$ & $T$\\
    $T$ & $F$ & $T$\\
    $T$ & $T$ & $F$\\
    \hline
  \end{tabular}
  \caption{Exercise 3.2}
\end{table}

\qs{}{
  4. Find a formula using only the connectives $\land$ and $\neg$ that is equivalent to $P \lor Q$.
  Justify your answer with a truth table.
}

\sol Using De Morgan's Laws we reach $\neg (\neg P \land \neg Q)$
  \begin{table}[h]
  \centering
  \label{tab:ex4 }
  \begin{tabular}{|c|c|c|c|}
    \hline
    $P$ & $Q$ & $(\neg P \land \neg Q)$ & $\neg (\neg P \land \neg Q)$\\
    \hline 
    $F$ & $F$ & $T$ & $F$\\   
    $F$ & $T$ & $F$ & $T$\\
    $T$ & $F$ & $F$ & $T$\\
    $T$ & $T$ & $F$ & $T$\\   
    \hline
  \end{tabular}
  \caption{Exercise 4}
\end{table}

\qs{}{
  *5. Some mathematicians use the symbol $\downarrow$ to mean nor. In other words,
  $P \downarrow Q$ means “neither $P$ nor $Q$.” 

  (a) Make a truth table for $P \downarrow Q$.

  (b) Find a formula using only the connectives $\land$, $\lor$, and $\neg$ that is equivalent to $P \downarrow Q$.

  (c) Find formulas using only the connective $\downarrow$ that are equivalent to $\neg P$, $P \lor Q$, and $P \land Q$.
}

\newpage

\sol \textbf{(a)} 
  \begin{table}[h]
  \centering
  \label{tab:ex5-1 }
  \begin{tabular}{|c|c|c|}
    \hline
    $P$ & $Q$ & $P \downarrow Q$\\
    \hline 
    $F$ & $F$ & $T$\\  
    $F$ & $T$ & $F$\\
    $T$ & $F$ & $F$\\ 
    $T$ & $T$ & $F$\\
    \hline
  \end{tabular}
  \caption{Exercise 5.1}
\end{table}

\sol \textbf{(b)} $\neg P \land \neg Q$ 
  \begin{table}[h]
  \centering
  \label{tab:ex5-2 }
  \begin{tabular}{|c|c|c|}
    \hline
    $P$ & $Q$ & $\neg P \land \neg Q$\\
    \hline 
    $F$ & $F$ & $T$\\  
    $F$ & $T$ & $F$\\
    $T$ & $F$ & $F$\\ 
    $T$ & $T$ & $F$\\
    \hline
  \end{tabular}
  \caption{Exercise 5.2}
\end{table}

\sol \textbf{(c)} \textbf{1} $P \downarrow P$, \textbf{2} $(P \downarrow Q) \downarrow (P \downarrow Q)$, \textbf{3} $(P \downarrow P) \downarrow (Q \downarrow Q)$ 
  \begin{table}[h]
    \centering
    \label{tab:ex5-3-1 }
    \begin{tabular}{|c|c|}
      \hline
      $P$ & $P \downarrow P$\\
      \hline 
      $F$ & $T$\\  
      $T$ & $F$\\
      \hline
    \end{tabular}
    \caption{Exercise 5.3.1}
  \end{table}

  \begin{table}[h]
    \centering
    \label{tab:ex5-3-2 }
    \begin{tabular}{|c|c|c|c|}
      \hline
      $P$ & $Q$ & $P \downarrow Q$ & $(P \downarrow Q) \downarrow (P \downarrow Q)$\\
      \hline 
      $F$ & $F$ & $T$ & $F$\\    
      $F$ & $T$ & $F$ & $T$\\
      $T$ & $F$ & $F$ & $T$\\
      $T$ & $T$ & $F$ & $T$\\ 
      \hline
    \end{tabular}
    \caption{Exercise 5.3.2}
  \end{table}

  \begin{table}[h]
    \centering
    \label{tab:ex5-3-3 }
    \begin{tabular}{|c|c|c|c|c|}
      \hline
      $P$ & $Q$ & $P \downarrow P$ & $Q \downarrow Q$ & $(P \downarrow P) \downarrow (Q \downarrow Q)$\\
      \hline 
      $F$ & $F$ & $T$ & $T$ & $F$\\   
      $F$ & $T$ & $T$ & $F$ & $F$\\
      $T$ & $F$ & $F$ & $T$ & $F$\\
      $T$ & $T$ & $F$ & $F$ & $T$\\    
      \hline
    \end{tabular}
    \caption{Exercise 5.3.3}
  \end{table}

\newpage

\qs{}{
}







