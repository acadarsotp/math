\section{Exercises}

\qs{}{
    *1. Analyze the logical forms of the following statements:

        (a) We’ll have either a reading assignment or homework problems, but we won’t have both homework problems and a test.

        (b) You won’t go skiing, or you will and there won’t be any snow.

        (c) $\sqrt{7} \nleq 2$
}

\sol \textbf{(a)} $(P \lor Q)  \land \neg(Q \land R)$ 

\sol \textbf{(b)} $(\neg P) \lor (P \land \neg Q)$

\sol \textbf{(c)} $\neg P \land \neg Q$

\qs{}{
    2. Analyze the logical forms of the following statements:

        (a) Either John and Bill are both telling the truth, or neither of them is.

        (b) I’ll have either fish or chicken, but I won’t have both fish and mashed potatoes.

        (c) 3 is a common divisor of 6, 9, and 15.  
}

\sol \textbf{(a)} $(P \land Q) \lor (\neg P \land \neg Q)$ 

\sol \textbf{(b)} $(P \lor Q) \land \neg (P \land R)$

\sol \textbf{(c)} $P \land Q \land R$

\newpage

\qs{}{
    3. Analyze the logical forms of the following statements:

        (a) Alice and Bob are not both in the room. 

        (b) Alice and Bob are both not in the room.

        (c) Either Alice or Bob is not in the room.

        (d) Neither Alice nor Bob is in the room.
}

\sol \textbf{(a)} $\neg (P \land Q)$

\sol \textbf{(b)} $\neg P \land \neg Q$

\sol \textbf{(c)} $\neg P \lor \neg Q$

\sol \textbf{(d)} $\neg P \land \neg Q$

\qs{}{
    4. Analyze the logical forms of the following statements:

        (a) Either both Ralph and Ed are tall, or both of them are handsome.

        (b) Both Ralph and Ed are either tall or handsome.

        (c) Both Ralph and Ed are neither tall nor handsome.

        (d) Neither Ralph nor Ed is both tall and handsome.
}

\sol \textbf{(a)} $(P \land Q) \lor (R \land S)$

\sol \textbf{(b)} $(P \lor R) \land (Q \land S)$

\sol \textbf{(c)} $(\neg P \land \neg R) \land (\neg Q \land \neg S)$

\sol \textbf{(d)} $\neg (P \land R) \land \neg (Q \land S)$

\qs{}{
    5. Which of the following expressions are well-formed formulas? 

        (a) $\neg (\neg P \lor \neg \neg Q)$.

        (b) $\neg (P, Q, \land R)$.

        (c) $P \land \neg P$.

        (d) $(P \land Q)(P \lor Q)$.
}

\sol \textbf{(b)} and \textbf{(d)} are not well-formed formulas.

\qs{}{
    *6. Let P stand for the statement “I will buy the pants” and S for the statement “I will buy the shirt.”
    What English sentences are represented by the following formulas?

        (a) $\neg (P \land \neg S)$.

        (b) $\neg P \land \neg S$.

        (c) $\neg P \lor \neg S$.
}

\sol \textbf{(a)} I won't buy the pants but not the shirt. 

\sol \textbf{(b)} I won't buy the pants and I won't buy the shirt.

\sol \textbf{(c)} I won't buy the pants or I won't buy the shirt.

\qs{}{
    7. Let S stand for the statement “Steve is happy” and G for “George is happy.” 
    What English sentences are represented by the following formulas?

        (a) $(S \lor G) \land (\neg S \lor \neg G)$.

        (b) $[S \lor (G \land \neg S)] \lor \neg G$.

        (c) $S \lor [G \land (\neg S \lor \neg G)]$.
}

\newpage

\sol \textbf{(a)} Either Steve or George is happy, but either Steve or George is not happy.

\sol \textbf{(b)} Either Steve is happy or Steve is not happy and George is happy, or George is not happy. 

\sol \textbf{(c)} Either Steve is happy, or George is happy but Steve is not happy or George is not happy.

\qs{}{
    8. Let T stand for the statement “Taxes will go up” and D for “The deficit will go up.”
    What English sentences are represented by the following formulas?

        (a) $T \lor D$.

        (b) $\neg (T \land D) \land \neg (\neg T \land \neg D)$.

        (c) $(T \land \neg D) \lor (D \land \neg T)$.
}

\sol \textbf{(a)} Taxes will go up or the deficit will go up.

\sol \textbf{(b)} Not both Taxes and deficit will go up, but not both taxes and deficit won't go up.

\sol \textbf{(c)} Either taxes will go up and the deficit will go down, or the deficit will go up and taxes will go down.

\qs{}{
    9. Identify the premises and conclusions of the following deductive arguments and analyze their logical forms. 
    Do you think the reasoning valid? 
    (Although you will have only your intuition to guide you in answering this last question, in the next section we will develop some techniques for determining the validity of arguments.)

        (a) Jane and Pete won’t both win the math prize. 
        Pete will win either the math prize or the chemistry prize. 
        Jane will win the math prize.
        Therefore, Pete will win the chemistry prize.

        (b) The main course will be either beef or fish.
        The vegetable will be either peas or corn.
        We will not have both fish as a main course and corn as a vegetable. 
        Therefore, we will not have both beef as a main course and peas as a vegetable.

        (c) Either John or Bill is telling the truth.
        Either Sam or Bill is lying.
        Therefore, either John is telling the truth or Sam is lying.

        (d) Either sales will go up and the boss will be happy, or expenses will go up and the boss won’t be happy. 
        Therefore, sales and expenses will not both go up.
}

\sol \textbf{(a)} 
    \begin{align*}
        \neg (J_m &\land P_m)\\
        P_m &\lor P_q\\ 
              &J_m\\
              &\therefore P_q
    \end{align*}
  
\sol \textbf{(b)} 
    \begin{align*}
        M_f &\lor M_b\\
        V_p &\lor V_c\\ 
        \neg (M_f &\land V_c)\\
        &\therefore \neg (M_b \land V_p) 
    \end{align*}
  
\sol \textbf{(c)}
    \begin{align*}
        J_t &\lor B_t\\
        \neg S_t &\lor \neg B_t\\ 
             &\therefore (J_t \lor \neg S_t)
    \end{align*}
  
\sol \textbf{(d)}
    \begin{align*}
        (S \land B) &\lor (E \land \neg B)\\
                    &\therefore \neg (S \land E)
    \end{align*}
  






