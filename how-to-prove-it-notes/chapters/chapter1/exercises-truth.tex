\subsection{Exercises}

\qs{}{
  3. In this exercise we will use the symbol + to mean exclusive or. 
  In other words, P + Q means “P or Q, but not both.”

  (a) Make a truth table for $P + Q$.

  (b) Find a formula using only the connectives $\land$, $\lor$, and $\neg$ that is
  equivalent to $P + Q$. Justify your answer with a truth table.
}

\sol \textbf{(a)} 
  \begin{table}[h]
  \centering
  \label{tab:ex3-1 }
  \begin{tabular}{|c|c|c|}
    \hline
    $P$ & $Q$ & $P + Q$\\ 
    \hline 
    $F$ & $F$ & $F$\\   
    $F$ & $T$ & $T$\\
    $T$ & $F$ & $T$\\
    $T$ & $T$ & $F$\\
    \hline
  \end{tabular}
  \caption{Exercise 3.1}
\end{table}

\sol \textbf{(b)} $(P \land \neg Q) \lor (\neg P \land Q)$ 
  \begin{table}[h]
  \centering
  \label{tab:ex3-2 }
  \begin{tabular}{|c|c|c|}
    \hline
    $P$ & $Q$ & $(P \land \neg Q) \lor (\neg P \land Q)$\\ 
    \hline 
    $F$ & $F$ & $F$\\   
    $F$ & $T$ & $T$\\
    $T$ & $F$ & $T$\\
    $T$ & $T$ & $F$\\
    \hline
  \end{tabular}
  \caption{Exercise 3.2}
\end{table}

\qs{}{
  4. Find a formula using only the connectives $\land$ and $\neg$ that is equivalent to $P \lor Q$.
  Justify your answer with a truth table.
}

\sol Using De Morgan's Laws we reach $\neg (\neg P \land \neg Q)$
  \begin{table}[h]
  \centering
  \label{tab:ex4 }
  \begin{tabular}{|c|c|c|c|}
    \hline
    $P$ & $Q$ & $(\neg P \land \neg Q)$ & $\neg (\neg P \land \neg Q)$\\
    \hline 
    $F$ & $F$ & $T$ & $F$\\   
    $F$ & $T$ & $F$ & $T$\\
    $T$ & $F$ & $F$ & $T$\\
    $T$ & $T$ & $F$ & $T$\\   
    \hline
  \end{tabular}
  \caption{Exercise 4}
\end{table}

\qs{}{
  *5. Some mathematicians use the symbol $\downarrow$ to mean nor. In other words,
  $P \downarrow Q$ means “neither $P$ nor $Q$.” 

  (a) Make a truth table for $P \downarrow Q$.

  (b) Find a formula using only the connectives $\land$, $\lor$, and $\neg$ that is equivalent to $P \downarrow Q$.

  (c) Find formulas using only the connective $\downarrow$ that are equivalent to $\neg P$, $P \lor Q$, and $P \land Q$.
}

\newpage

\sol \textbf{(a)} 
  \begin{table}[h]
  \centering
  \label{tab:ex5-1 }
  \begin{tabular}{|c|c|c|}
    \hline
    $P$ & $Q$ & $P \downarrow Q$\\
    \hline 
    $F$ & $F$ & $T$\\  
    $F$ & $T$ & $F$\\
    $T$ & $F$ & $F$\\ 
    $T$ & $T$ & $F$\\
    \hline
  \end{tabular}
  \caption{Exercise 5.1}
\end{table}

\sol \textbf{(b)} $\neg P \land \neg Q$ 
  \begin{table}[h]
  \centering
  \label{tab:ex5-2 }
  \begin{tabular}{|c|c|c|}
    \hline
    $P$ & $Q$ & $\neg P \land \neg Q$\\
    \hline 
    $F$ & $F$ & $T$\\  
    $F$ & $T$ & $F$\\
    $T$ & $F$ & $F$\\ 
    $T$ & $T$ & $F$\\
    \hline
  \end{tabular}
  \caption{Exercise 5.2}
\end{table}

\sol \textbf{(c)} \textbf{1} $P \downarrow P$, \textbf{2} $(P \downarrow Q) \downarrow (P \downarrow Q)$, \textbf{3} $(P \downarrow P) \downarrow (Q \downarrow Q)$ 
  \begin{table}[h]
    \centering
    \label{tab:ex5-3-1 }
    \begin{tabular}{|c|c|}
      \hline
      $P$ & $P \downarrow P$\\
      \hline 
      $F$ & $T$\\  
      $T$ & $F$\\
      \hline
    \end{tabular}
    \caption{Exercise 5.3.1}
  \end{table}

  \begin{table}[h]
    \centering
    \label{tab:ex5-3-2 }
    \begin{tabular}{|c|c|c|c|}
      \hline
      $P$ & $Q$ & $P \downarrow Q$ & $(P \downarrow Q) \downarrow (P \downarrow Q)$\\
      \hline 
      $F$ & $F$ & $T$ & $F$\\    
      $F$ & $T$ & $F$ & $T$\\
      $T$ & $F$ & $F$ & $T$\\
      $T$ & $T$ & $F$ & $T$\\ 
      \hline
    \end{tabular}
    \caption{Exercise 5.3.2}
  \end{table}

  \begin{table}[h]
    \centering
    \label{tab:ex5-3-3 }
    \begin{tabular}{|c|c|c|c|c|}
      \hline
      $P$ & $Q$ & $P \downarrow P$ & $Q \downarrow Q$ & $(P \downarrow P) \downarrow (Q \downarrow Q)$\\
      \hline 
      $F$ & $F$ & $T$ & $T$ & $F$\\   
      $F$ & $T$ & $T$ & $F$ & $F$\\
      $T$ & $F$ & $F$ & $T$ & $F$\\
      $T$ & $T$ & $F$ & $F$ & $T$\\    
      \hline
    \end{tabular}
    \caption{Exercise 5.3.3}
  \end{table}

\newpage

\qs{}{
}
